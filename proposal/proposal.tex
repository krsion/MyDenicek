\documentclass{article}

\usepackage[a-2u]{pdfx}
\usepackage[utf8]{inputenc}
\usepackage{array}
\usepackage[ddmmyyyy]{datetime}
\usepackage{geometry}

\renewcommand{\dateseparator}{.}
\renewcommand{\baselinestretch}{1.5}
\geometry{margin=2cm}

\def\ResearchProject{Research project}
\def\CompanyProject{Company project}
\def\SoftwareProject{Software project}

% METADATA SECTION START
\def\StudentsFullName{Bc. Ondřej Krsička}
\def\Obor{Computer Science - Software and Data Engineering}
\def\ProjetTitle{Using CRDTs to enable collaborative editing in Denicek}
\def\ProjectType{\ResearchProject}
\def\SupervisorFullName{Mgr. Tomáš Petříček, Ph.D.}
\def\ExpectedStart{1.11.2025}
\def\ExpectedEnd{1.7.2026}
% METADATA SECTION END

\title{\ProjetTitle}
\author{\StudentsFullName}

\begin{document}

\centerline{\Large \textbf{Proposal of team software project}}
\centerline{ \textbf{Department of Software Engineering}}
\centerline{ \textbf{Faculty of Mathematics and Physics, Charles University}}
\bigskip
{\noindent\begin{tabular*}{\textwidth}{ >{\raggedleft}m{4cm} l}
 {\bf Solvers:} & \StudentsFullName \\
 {\bf Study program:} & \Obor \\
 & \\
 {\bf Project title:} & \ProjetTitle \\
 {\bf Project type:} & \ProjectType \\
 & \\
 {\bf Supervisor:} & \SupervisorFullName \\
 & \\
 {\bf Expected start:} & \ExpectedStart \\
 {\bf Expected end:} & \ExpectedEnd \\
\end{tabular*}}


\section{Introduction}
\emph{Denicek} \cite{denicek2024} is a  document-based end-user programming substrate and \emph{Webnicek} is a web application built on top of \emph{Denicek}. \emph{Denicek} synchronization is done using Operational Transformation, which is error prone and complex and requires a central server for synchronization. This project will create \emph{MyDenicek} backend built on top of CRDTs to make it simpler to implement and to follow principles of local-first software. \emph{MyWebnicek} will be a new web-application on top of the new substrate.
\section{Denicek end-user programming experiences}
\emph{Denicek: Computational Substrate for Document-Oriented End-User Programming} \cite{denicek2024} provides the following end-user programming experiences:
\begin{itemize}
\setlength{\itemsep}{-5pt}
\item Collaborative editing
\item Programming by demonstration (users can record actions and program a button to replay them on click)
\item Incremental recomputation (formulas depending on changed values are invalidated and automatically recomputed)
\item Schema change control (when a value is wrapped or unwrapped in an element, all references to it are updated)
\item End-user debugging (values on which a formula depends can be highlighted for better understanding of the result)
\item Concrete programming (copying and pasting formulas between contexts while preserving dependencies; when the original changes, the copied formula updates accordingly)
\end{itemize}

\section{Denicek document structure}
A \emph{Denicek document} is a tree composed of named nodes of the following kinds:
\begin{itemize}
\setlength{\itemsep}{-5pt}
\item Ordered lists of nodes: all items are of the same type and addressable by index
\item Records: children can be of different types and are addressable by their name
\item References to other locations in the document
\item Primitives: numeric or textual values
\end{itemize}

\section{Denicek programs}
A \emph{Denicek program} is represented as a history of primitive actions. Histories can be replayed, merged, and compared, with conflict detection between them.

Primitive actions that do not affect references include:
\begin{itemize}
\setlength{\itemsep}{-5pt}
\item Add to record
\item Append to list
\item Reorder list
\item Delete item from list
\item Update tag of list or record
\item Edit a primitive value (text or number)
\end{itemize}

Primitive actions that \emph{do} affect references include:
\begin{itemize}
\setlength{\itemsep}{-5pt}
\item Rename a record
\item Delete a record
\item Wrap a node in a record
\item Wrap a node in a list
\item Copy node(s) from given selectors to specified target(s)
\end{itemize}


\section{Denicek usage}
Two end-user programming environments are currently built on top of the \emph{Denicek} substrate:
\begin{itemize}
\setlength{\itemsep}{-5pt}
\item \emph{Webnicek}: for creating interactive HTML documents
\item \emph{Datnicek}: for data science use cases similar to Jupyter Notebooks
\end{itemize}

The use cases of Denicek are illustrated by the following formative examples:
\begin{itemize}
\setlength{\itemsep}{-5pt}
\item \textbf{Counter app:} The user creates a document with value 1, wraps it in a formula (1+1), records the process, and replays it on button click: producing a simple counter.
\item \textbf{Todo app:} Similar to the counter app, but the recorded actions add text from an input field as a new list item.
\item \textbf{Conference list:} Alice refactors a list of speakers from comma-separated items (name, email) into a table with columns (name, email). Meanwhile, Bob adds a new list item. The merged result is a table containing all speakers, including Bob’s addition.
\item \textbf{Conference budget:} Building on the conference list, formulas depending on its values are automatically recomputed during refactoring.
\item \textbf{Hello world:} The user makes the first line of a list of sentences lowercase, then capitalizes the first letter, copies the formula, and applies it to the whole list.
\item \textbf{Traffic accidents:} Omitted from this research project, as it relates more closely to \emph{Datnicek}.
\end{itemize}


\section{Limitations of Denicek and goals of this research project}
The current \emph{Denicek} synchronization layer is implemented using Operational Transformation (OT). However, OT is complex, error-prone, and requires a central synchronization server, which does not satisfy the requirements of local-first software \cite{klep2019}.

The goal of this research project is to create an alternative to \emph{Webnicek} built on top of a CRDT-based local-first substrate. The new system should support the same end-user programming experiences demonstrated by the formative examples, possibly using a revised set of primitive operations. It should also maintain a clear separation between the user interface and the CRDT substrate to enable future development of a \emph{Datnicek} alternative based on the same backend.

The resulting systems will be called \emph{MyDenicek} and \emph{MyWebnicek}, inspired by \emph{MyWebstrates} \cite{klokmose2024}: a local-first, CRDT-based alternative to Webstrates \cite{klokmose2015}. Both parts will be implemented in TypeScript.

\subsection{MyWebnicek functional requirements}
\emph{MyWebnicek} will have the following functionalities:
\begin{itemize}
\setlength{\itemsep}{-5pt}
    \item Renderer of the final document
    \item Navigation through the document
    \item Commandline for user to perform primitive actions
    \item History view of user edit actions
    \item Sharing mechanism for collaborative editing via network
    \item Conflict resolution interface
    
\end{itemize}

% \section{Project description}

% The project will initially explore using a representation based on \emph{Grove}~\cite{grove2024} as the underlying replication model. Grove was designed as the basis for collaborative structure editor for program editing. It represents documents as Groves (labelled directed multi-graph). User edits in Grove correspond to edit actions that amount to edge insertion and deletion in the underlying graph.
% %
% We will design a representation of \emph{Denicek} documents as Groves and explore ways of encoding \emph{Denicek} edit operations as operations on the underlying Grove graph structure. This aims to leverage Grove's principled patch composition for correctness and maintainability.

% The design will not be limited to Grove: the project will remain open to exploring alternative CRDT approaches (e.g., operation-based, state-based, and delta-state-based CRDTs) or adopting existing mature frameworks such as Yjs and Automerge. The chosen approach for the final integration will be based on performance, integration complexity, implementation complexity, and user experience.

% \subsection{Project tasks}
% The project will consist of the following main tasks:

% \begin{enumerate}
%     \item \textbf{Grove-based document modeling:}
%     Model Denicek documents as Groves (labelled directed multi-graph). This will require designing suitable encoding for Denicek document aspects not currently handled in Grove (including HTML-like element tag names, oredered lists of child nodes).

%     \item \textbf{Grove-based edit mapping:}
%     Map user edit actions from Denicek (add, append, delete, rename, wrap, etc.) to Grove edit actions and/or Grove patch language constructs. Grove does not currently support copy operation (only relocation), which will require a novel approach.

%     \item \textbf{Selection and evaluation of alternative approaches:}
%     Compare Grove with alternative CRDT approaches and frameworks (state-based CRDTs, delta-state-based CRDTs, operation-based CRDTs; Yjs, Automerge). Criteria include: expressiveness for structured documents, merge semantics and locality of edits, support for offline-first and partial replication, performance under contention, storage/GC behavior, and integration complexity with the existing stack.

%     \item \textbf{Prototype implementation:}
%     Implement a CRDT-backed synchronization layer for \emph{Denicek}, based either on the Grove prototype or, possibly, using a suitable widely adopted framework (e.g., Yjs). The prototype should be capable of running the six formative examples used in the Denicek design \cite{denicek2024}. The prototype will, additionally, support local first document editing, real-time propagation, and reconciliation after partitions.

%     \item \textbf{Testing and evaluation:}
%     %Conduct reproducible experiments measuring latency, throughput, and memory footprint under realistic collaboration patterns (multi-user burst edits, partitions, concurrent structural edits).
%     Evaluate correctness (e.g., convergence, intention preservation for structured edits), and developer ergonomics (API surface and debugging). Compare user experience against the existing OT deployment via small-scale user testing.

%     \item \textbf{Documentation and dissemination:}
%     Prepare a technical report summarizing the Grove mapping, alternative CRDT evaluations, benchmarks, limitations, and recommendations for Denicek's long-term architecture. The resulting new design will be described in paper submitted to a suitable conference or a workshop.
% \end{enumerate}

% \noindent
% By introducing CRDTs:and specifically evaluating a typed, patch-based approach via Grove:into Denicek, the platform can achieve robust offline collaboration, principled structural merges, and simplified synchronization logic.

\section{Schedule}
The project is expected to run for 8 months. The expected schedule is:

\begin{itemize}
\setlength{\itemsep}{-5pt}
\item Task 1 (Studying of materials, analysis and prototyping) - Month 1-3
\item Task 2 (Production-ready MyDenicek and MyWebnicek implementation) - Month 4-7
\item Task 3 (Textual and Video Documentation) - Month 8
\item Task 4 (Testing and evaluation) - Month 8
\end{itemize}

% \noindent
% The project will deliver:

% \begin{itemize}
%   \setlength{\itemsep}{-5pt}
%   \item A working CRDT-based synchronization backend for Denicek.
%   \item A comparative evaluation of CRDT base approach versus currently used Operational Transformation (OT)  for Denicek's structured documents and end-user programming context.
%   \item Technical report describing the implementaiton and evaluation
% \end{itemize}

\section{Team structure}
The work will be coordinated by the supervisor. The student will be responsible for the technical aspects of the project (developing the Grove encoding, prototype implementation) and evaluation. The design of the system will be developed in collaboration between the supervisor, the student and Jonathan Edwards (Denicek co-author). 

\section{Related work}
This work directly builds on the Denicek system presented at \textit{ACM UIST 2025}~\cite{denicek2024}. It extends ongoing research on Conflict-free Replicated Data Types (CRDTs), as summarized in Preguiça’s overview~\cite{crdtoverview}, and is informed by practical open-source CRDT frameworks (Yjs, Automerge). The project will first investigate the applicability of Grove’s typed, patch-based model~\cite{grove2024} to Denicek’s document structures and editing workflows, while keeping the option to adopt or hybridize with other CRDTs if Grove proves unsuitable for some features.



\bibliographystyle{plain}
\begin{thebibliography}{9}
\bibitem{denicek2024}
T. Petříček, et al. \textit{Denicek: Computational Substrate for Document-Oriented End-User Programming.} In Proceedings of the 38th Annual ACM Symposium on User Interface Software and Technology (UIST '25). no. 32, pp. 1--19.
\bibitem{crdtoverview}
N. Preguiça. \textit{Conflict-free Replicated Data Types: An Overview.} 2018. \url{https://arxiv.org/abs/1806.10254}.
\bibitem{grove2024}
% Update authors/year/venue if you maintain a .bib file
M. D. Adams, et al. \textit{Grove: A Bidirectionally Typed Collaborative Structure Editor Calculus.} ACM, 2024. DOI: \url{https://doi.org/10.1145/3704909}.
\bibitem{cypher1993}
A. Cypher, et al. (eds.) Watch what I do: programming by demonstration. MIT Press, 1993.
\bibitem{edwards2024}
J. Edwards, et al. Schema Evolution in Interactive Programming Systems. The Art, Science, and Engineering of Programming, 9(1), 2-1. 2024.
\bibitem{klep2019}
M. Kleppmann, et al. Local-first software: you own your data, in spite of the cloud. Proceedings of the 2019 ACM SIGPLAN International Symposium on New Ideas, New Paradigms, and Reflections on Programming and Software. 2019.
\bibitem{klokmose2024}
Clemens Nylandsted Klokmose, James R. Eagan, and Peter van Hardenberg. 2024. MyWebstrates: Webstrates as Local-first Software. In Proceedings of the 37th Annual ACM Symposium on User Interface Software and Technology (UIST '24). Association for Computing Machinery, New York, NY, USA, Article 42, 1–12. https://doi.org/10.1145/3654777.3676445
\bibitem{klokmose2015}
Clemens N. Klokmose, James R. Eagan, Siemen Baader, Wendy Mackay, and Michel Beaudouin-Lafon. 2015. Webstrates: Shareable Dynamic Media. In Proceedings of the 28th Annual ACM Symposium on User Interface Software \& Technology (UIST '15). Association for Computing Machinery, New York, NY, USA, 280–290. https://doi.org/10.1145/2807442.2807446

\end{thebibliography}
\end{document}
